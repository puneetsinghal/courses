\newcommand{\duedate}{7 November 2016, 23:59}
\documentclass{/home/puneet/Desktop/courses/16384/Assignments/16384_doc}
\newcommand{\assignmentname}{Milestone: 3D Feeding Robot}
\newcommand{\comment}[1]{}
\usepackage{url}

\newcommand{\feedbackURL}{https://goo.gl/hKFctm}

\ifdef{\writeup}{\printanswers}

\begin{document}
\maketitle

\section{Overview}
In your capstone project, you are asked to adapt the 2D Robot.m class you wrote before to a new 3D version.

\begin{itemize}
\item Start by creating a copy of Robot.m file and name the file "Robot3D.m"

\item Use a matrix of DH parameters as the argument for the class constructor,
rather than just link lengths. You do not need to worry about mass information,
but if you end up adding gravity compensation torques to improve performance
later you will also want the center of mass and actual mass for each link and joint as well as for the end effector.

\item Your Robot class should be able to adapt to a general n-link robot. (You can use the information from DH parameters to initialize parameters like degrees of freedom. 

\item Modify forward\_kinematics method of robot class to return the frames, given the joint angles as argument.

\item Modify end\_effector method of robot class to return the end effector description \smcolvec{x\\ y\\ z\\ roll\\ pitch\\ yaw}, given the joint angles as argument

\item Modify the jacobians method to calculate the Jacobian for the manipulator arm
\end{itemize}

\section{Variables}
For testing, we have provided you with a test file named test\_robot\_class.m in the milestone\_handout folder. Copy your Robot3D.m file inside milestone\_handout folder. Assume that the robot is weightless and only forces acting on the robot are at end effector. 

The test\_data.mat file provided to you contains the following variables:
\begin{itemize}
\item DH\_parameter: DH parameter for 6 dof arm
\item theta: joint angle trajectory with dimension as $6\times n$, where 6 is the number of joints and $n$ is the number of points on trajectory
\item position\_truth: this variable contains the truth value for end effector position for given joint angles. Column represents \smcolvec{x\\ y\\ z\\ roll\\ pitch\\ yaw}
\item joint\_torques:  joint torques trajectory with dimension as $6\times n$, where 6 is the number of joints and $n$ is the number of points on trajectory
\item forces\_truth: this variable contains the truth value for end effector forces (forces applied on end effector) for given joint angles. Column represents \smcolvec{F_x\\ F_y\\ F_z\\ M_x\\ M_y\\ M_z}
\end{itemize}


\section{Verification}
Modify the test\_robot\_class.m file to create an instance of robot class and calculate the position of end effector \smcolvec{x\\ y\\ z\\ roll\\ pitch\\ yaw} and forces on end effector \smcolvec{F_x\\ F_y\\ F_z\\ M_x\\ M_y\\ M_z} for each joint angle. Run the setup file and then run test\_robot\_class. Both requires no argument. You will get two plots comparing your values of end effector position and forces with ground truth. 

\begin{submissionChecklist}
		\item Coding questions:
    	\begin{checklist}
        	\item Run \verb!create_submission.m! in Matlab.
        	\item Upload \verb!handin-6.tar.gz! to Autolab.
    	\end{checklist}
	\end{submissionChecklist}

\end{document}

\end{document}
